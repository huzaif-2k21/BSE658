% Options for packages loaded elsewhere
\PassOptionsToPackage{unicode}{hyperref}
\PassOptionsToPackage{hyphens}{url}
%
\documentclass[
]{article}
\usepackage{amsmath,amssymb}
\usepackage{iftex}
\ifPDFTeX
  \usepackage[T1]{fontenc}
  \usepackage[utf8]{inputenc}
  \usepackage{textcomp} % provide euro and other symbols
\else % if luatex or xetex
  \usepackage{unicode-math} % this also loads fontspec
  \defaultfontfeatures{Scale=MatchLowercase}
  \defaultfontfeatures[\rmfamily]{Ligatures=TeX,Scale=1}
\fi
\usepackage{lmodern}
\ifPDFTeX\else
  % xetex/luatex font selection
\fi
% Use upquote if available, for straight quotes in verbatim environments
\IfFileExists{upquote.sty}{\usepackage{upquote}}{}
\IfFileExists{microtype.sty}{% use microtype if available
  \usepackage[]{microtype}
  \UseMicrotypeSet[protrusion]{basicmath} % disable protrusion for tt fonts
}{}
\makeatletter
\@ifundefined{KOMAClassName}{% if non-KOMA class
  \IfFileExists{parskip.sty}{%
    \usepackage{parskip}
  }{% else
    \setlength{\parindent}{0pt}
    \setlength{\parskip}{6pt plus 2pt minus 1pt}}
}{% if KOMA class
  \KOMAoptions{parskip=half}}
\makeatother
\usepackage{xcolor}
\usepackage[margin=1in]{geometry}
\usepackage{color}
\usepackage{fancyvrb}
\newcommand{\VerbBar}{|}
\newcommand{\VERB}{\Verb[commandchars=\\\{\}]}
\DefineVerbatimEnvironment{Highlighting}{Verbatim}{commandchars=\\\{\}}
% Add ',fontsize=\small' for more characters per line
\usepackage{framed}
\definecolor{shadecolor}{RGB}{248,248,248}
\newenvironment{Shaded}{\begin{snugshade}}{\end{snugshade}}
\newcommand{\AlertTok}[1]{\textcolor[rgb]{0.94,0.16,0.16}{#1}}
\newcommand{\AnnotationTok}[1]{\textcolor[rgb]{0.56,0.35,0.01}{\textbf{\textit{#1}}}}
\newcommand{\AttributeTok}[1]{\textcolor[rgb]{0.13,0.29,0.53}{#1}}
\newcommand{\BaseNTok}[1]{\textcolor[rgb]{0.00,0.00,0.81}{#1}}
\newcommand{\BuiltInTok}[1]{#1}
\newcommand{\CharTok}[1]{\textcolor[rgb]{0.31,0.60,0.02}{#1}}
\newcommand{\CommentTok}[1]{\textcolor[rgb]{0.56,0.35,0.01}{\textit{#1}}}
\newcommand{\CommentVarTok}[1]{\textcolor[rgb]{0.56,0.35,0.01}{\textbf{\textit{#1}}}}
\newcommand{\ConstantTok}[1]{\textcolor[rgb]{0.56,0.35,0.01}{#1}}
\newcommand{\ControlFlowTok}[1]{\textcolor[rgb]{0.13,0.29,0.53}{\textbf{#1}}}
\newcommand{\DataTypeTok}[1]{\textcolor[rgb]{0.13,0.29,0.53}{#1}}
\newcommand{\DecValTok}[1]{\textcolor[rgb]{0.00,0.00,0.81}{#1}}
\newcommand{\DocumentationTok}[1]{\textcolor[rgb]{0.56,0.35,0.01}{\textbf{\textit{#1}}}}
\newcommand{\ErrorTok}[1]{\textcolor[rgb]{0.64,0.00,0.00}{\textbf{#1}}}
\newcommand{\ExtensionTok}[1]{#1}
\newcommand{\FloatTok}[1]{\textcolor[rgb]{0.00,0.00,0.81}{#1}}
\newcommand{\FunctionTok}[1]{\textcolor[rgb]{0.13,0.29,0.53}{\textbf{#1}}}
\newcommand{\ImportTok}[1]{#1}
\newcommand{\InformationTok}[1]{\textcolor[rgb]{0.56,0.35,0.01}{\textbf{\textit{#1}}}}
\newcommand{\KeywordTok}[1]{\textcolor[rgb]{0.13,0.29,0.53}{\textbf{#1}}}
\newcommand{\NormalTok}[1]{#1}
\newcommand{\OperatorTok}[1]{\textcolor[rgb]{0.81,0.36,0.00}{\textbf{#1}}}
\newcommand{\OtherTok}[1]{\textcolor[rgb]{0.56,0.35,0.01}{#1}}
\newcommand{\PreprocessorTok}[1]{\textcolor[rgb]{0.56,0.35,0.01}{\textit{#1}}}
\newcommand{\RegionMarkerTok}[1]{#1}
\newcommand{\SpecialCharTok}[1]{\textcolor[rgb]{0.81,0.36,0.00}{\textbf{#1}}}
\newcommand{\SpecialStringTok}[1]{\textcolor[rgb]{0.31,0.60,0.02}{#1}}
\newcommand{\StringTok}[1]{\textcolor[rgb]{0.31,0.60,0.02}{#1}}
\newcommand{\VariableTok}[1]{\textcolor[rgb]{0.00,0.00,0.00}{#1}}
\newcommand{\VerbatimStringTok}[1]{\textcolor[rgb]{0.31,0.60,0.02}{#1}}
\newcommand{\WarningTok}[1]{\textcolor[rgb]{0.56,0.35,0.01}{\textbf{\textit{#1}}}}
\usepackage{graphicx}
\makeatletter
\def\maxwidth{\ifdim\Gin@nat@width>\linewidth\linewidth\else\Gin@nat@width\fi}
\def\maxheight{\ifdim\Gin@nat@height>\textheight\textheight\else\Gin@nat@height\fi}
\makeatother
% Scale images if necessary, so that they will not overflow the page
% margins by default, and it is still possible to overwrite the defaults
% using explicit options in \includegraphics[width, height, ...]{}
\setkeys{Gin}{width=\maxwidth,height=\maxheight,keepaspectratio}
% Set default figure placement to htbp
\makeatletter
\def\fps@figure{htbp}
\makeatother
\setlength{\emergencystretch}{3em} % prevent overfull lines
\providecommand{\tightlist}{%
  \setlength{\itemsep}{0pt}\setlength{\parskip}{0pt}}
\setcounter{secnumdepth}{-\maxdimen} % remove section numbering
\ifLuaTeX
  \usepackage{selnolig}  % disable illegal ligatures
\fi
\usepackage{bookmark}
\IfFileExists{xurl.sty}{\usepackage{xurl}}{} % add URL line breaks if available
\urlstyle{same}
\hypersetup{
  pdftitle={Descriptive Statistics: Scaling and Correlations},
  hidelinks,
  pdfcreator={LaTeX via pandoc}}

\title{Descriptive Statistics: Scaling and Correlations}
\author{}
\date{\vspace{-2.5em}}

\begin{document}
\maketitle

After taking a first look at our data in the last notebook, now we want
to start looking at it more closely as per our needs and requirements.

\paragraph{Scaling}\label{scaling}

In simple terms, scaling refers to changing size of an object without
affecting its shape.

\subparagraph{Linear Transformation:}\label{linear-transformation}

A linear transformation involves addition, subtraction, multiplication,
or division with a constant value. For example, if you add 1 to the
numbers 2, 4, and 6, the resulting numbers (3, 5, and 7) are a linear
transformation of the original numbers. Linear transformations are
useful, because they allow you to represent your data in a metric that
is suitable to you and your audience.

\textbf{Centering:}

`Centering' is a particularly common linear transformation. This linear
transfor- mation is frequently applied to continuous predictor
variables. To center a predictor variable, subtract the mean of that
predictor variable from each data point. As a result, each data point is
expressed in terms of how much it is above the mean (positive score) or
below the mean (negative score). Thus, subtracting the mean out of the
variable expresses each data point as a mean-deviation score. The value
zero now has a new meaning for this variable: it is at the `center' of
the variable's distribution, namely, the mean.

\textbf{Standardizing:}

A second common linear transformation is `standardizing' or
`z--scoring'. For standardizing, the centered variable is divided by the
standard deviation of the sample.

Let's look at an example:

The following are response durations from a psycholinguistic experiment:

\texttt{460ms\ 480ms\ 500ms\ 520ms\ 540ms}

The mean of these five numbers is \texttt{500ms}.

Centering these numbers results in the following:

\texttt{−\ 40ms\ −\ 20ms\ 0ms\ +20ms\ +\ 40ms}

The standard deviation (learnt in last notebook) for these numbers is
\texttt{\textasciitilde{}32ms}.

To `standardize', we have to divide the centered data by the standard
deviation. For example, the first point, \texttt{–40ms}, divided by
\texttt{32ms}, yields \texttt{–1.3}. Since each data point is divided by
the same number, this change qualifies as a linear transformation.

As a result of standardization, you get the following numbers (rounded
to one digit):

\texttt{−1.3z\ −\ 0.6z\ 0z\ +\ 0.6z\ +1.3z}

The raw response duration \texttt{460ms} is \texttt{–40ms} (after
centering), which corresponds to being \texttt{1.3} standard deviations
below the mean. Thus, standardization involves re-expressing the data in
terms of \textbf{how many standard deviations they are away from the
mean}.

\subparagraph{But why this extra
effort?}\label{but-why-this-extra-effort}

Standardizing is a way of getting rid of a variable's metric. In a
situation with multiple variables, each variable may have a different
standard deviation, but by dividing each variable by the respective
standard deviation, it is possible to convert all variables into a scale
of \textbf{standard units}. This sometimes may help in making variables
comparable, for example, when assessing the relative impact of multiple
predictors. For example, if you can imagine we have two questionnaires -
one for extraversion where you scored 2 out of 10 and the other for
grumpiness where you scored 35 out of 50, then it doesn't make a lot of
sense to try to compare your raw score of 2 on the extraversion
questionnaire to your raw score of 35 on the grumpiness questionnaire.
The raw scores for the two variables are ``about'' fundamentally
different things, so this would be like comparing apples to oranges. But
if you standardize them, they will still become comparable in some
sense.

Let's also examine the score of 35 out of 50 for grumpiness. Would this
mean that you're 70\% grumpy? Instead of interpreting raw data this way,
it would make more sense if we describe your grumpiness in terms of the
overall distribution of the grumpiness of humans which is possible
through standardisation i.e.~where do you lie on the grumpiness spectrum
of the all humans? ;)

\begin{Shaded}
\begin{Highlighting}[]
\CommentTok{\#Try it out yourself}
\CommentTok{\#Define a vector with Grumpiness scores of you and your friends and find the z score for your self}
\NormalTok{X }\OtherTok{=} \FunctionTok{c}\NormalTok{(}\DecValTok{60}\NormalTok{,}\DecValTok{72}\NormalTok{,}\DecValTok{89}\NormalTok{,}\DecValTok{113}\NormalTok{,}\DecValTok{137}\NormalTok{,}\DecValTok{170}\NormalTok{)                       }
\NormalTok{z }\OtherTok{=}\NormalTok{ (X }\SpecialCharTok{{-}} \FunctionTok{mean}\NormalTok{(X)) }\SpecialCharTok{/} \FunctionTok{sd}\NormalTok{(X)}
\NormalTok{z}
\end{Highlighting}
\end{Shaded}

\begin{verbatim}
## [1] -1.1251491 -0.8368547 -0.4284376  0.1481513  0.7247402  1.5175499
\end{verbatim}

\emph{Reference: Chapter 5, Winter B.}

\paragraph{Correlation}\label{correlation}

So far we have focused entirely on how to construct descriptive
statistics for a single variable. We haven't talked about how to
describe the relationships between variables in the data. To do that, we
want to talk mostly about the correlation between variables.

\begin{Shaded}
\begin{Highlighting}[]
\FunctionTok{setwd}\NormalTok{(}\StringTok{"C:/Users/hp/Documents/GitHub/BSE658/Module 3/Notebooks/"}\NormalTok{)}
\FunctionTok{library}\NormalTok{(lsr)}
\FunctionTok{library}\NormalTok{(psych)}
\FunctionTok{library}\NormalTok{(ggplot2)}
\end{Highlighting}
\end{Shaded}

\begin{verbatim}
## 
## Attaching package: 'ggplot2'
\end{verbatim}

\begin{verbatim}
## The following objects are masked from 'package:psych':
## 
##     %+%, alpha
\end{verbatim}

\begin{Shaded}
\begin{Highlighting}[]
\FunctionTok{library}\NormalTok{(tidyverse)}
\end{Highlighting}
\end{Shaded}

\begin{verbatim}
## -- Attaching core tidyverse packages ------------------------ tidyverse 2.0.0 --
## v dplyr     1.1.4     v readr     2.1.5
## v forcats   1.0.0     v stringr   1.5.1
## v lubridate 1.9.3     v tibble    3.2.1
## v purrr     1.0.2     v tidyr     1.3.1
\end{verbatim}

\begin{verbatim}
## -- Conflicts ------------------------------------------ tidyverse_conflicts() --
## x ggplot2::%+%()   masks psych::%+%()
## x ggplot2::alpha() masks psych::alpha()
## x dplyr::filter()  masks stats::filter()
## x dplyr::lag()     masks stats::lag()
## i Use the conflicted package (<http://conflicted.r-lib.org/>) to force all conflicts to become errors
\end{verbatim}

\begin{Shaded}
\begin{Highlighting}[]
\CommentTok{\#Let\textquotesingle{}s load some data}
\FunctionTok{load}\NormalTok{( }\StringTok{"parenthood.Rdata"}\NormalTok{ )}
\FunctionTok{head}\NormalTok{(parenthood)}
\end{Highlighting}
\end{Shaded}

\begin{verbatim}
##   dan.sleep baby.sleep dan.grump day
## 1      7.59      10.18        56   1
## 2      7.91      11.66        60   2
## 3      5.14       7.92        82   3
## 4      7.71       9.61        55   4
## 5      6.68       9.75        67   5
## 6      5.99       5.04        72   6
\end{verbatim}

\begin{Shaded}
\begin{Highlighting}[]
\FunctionTok{dim}\NormalTok{(parenthood)}
\end{Highlighting}
\end{Shaded}

\begin{verbatim}
## [1] 100   4
\end{verbatim}

\begin{Shaded}
\begin{Highlighting}[]
\CommentTok{\#Try describe() for the above dataframe}
\FunctionTok{describe}\NormalTok{(parenthood)}
\end{Highlighting}
\end{Shaded}

\begin{verbatim}
##            vars   n  mean    sd median trimmed   mad   min    max range  skew
## dan.sleep     1 100  6.97  1.02   7.03    7.00  1.09  4.84   9.00  4.16 -0.29
## baby.sleep    2 100  8.05  2.07   7.95    8.05  2.33  3.25  12.07  8.82 -0.02
## dan.grump     3 100 63.71 10.05  62.00   63.16  9.64 41.00  91.00 50.00  0.43
## day           4 100 50.50 29.01  50.50   50.50 37.06  1.00 100.00 99.00  0.00
##            kurtosis   se
## dan.sleep     -0.72 0.10
## baby.sleep    -0.69 0.21
## dan.grump     -0.16 1.00
## day           -1.24 2.90
\end{verbatim}

\begin{Shaded}
\begin{Highlighting}[]
\CommentTok{\#Let\textquotesingle{}s also take a graphical look at the data }
\FunctionTok{hist}\NormalTok{(parenthood}\SpecialCharTok{$}\NormalTok{dan.sleep)}
\end{Highlighting}
\end{Shaded}

\includegraphics{Module3_Nb2_files/figure-latex/unnamed-chunk-4-1.pdf}

\begin{Shaded}
\begin{Highlighting}[]
\CommentTok{\#Try plotting for the other 2 variables}
\FunctionTok{hist}\NormalTok{(parenthood}\SpecialCharTok{$}\NormalTok{dan.grump)}
\end{Highlighting}
\end{Shaded}

\includegraphics{Module3_Nb2_files/figure-latex/unnamed-chunk-4-2.pdf}

\begin{Shaded}
\begin{Highlighting}[]
\FunctionTok{hist}\NormalTok{(parenthood}\SpecialCharTok{$}\NormalTok{baby.sleep)}
\end{Highlighting}
\end{Shaded}

\includegraphics{Module3_Nb2_files/figure-latex/unnamed-chunk-4-3.pdf}

\begin{Shaded}
\begin{Highlighting}[]
\FunctionTok{hist}\NormalTok{(parenthood}\SpecialCharTok{$}\NormalTok{day)}
\end{Highlighting}
\end{Shaded}

\includegraphics{Module3_Nb2_files/figure-latex/unnamed-chunk-4-4.pdf}

But we now want to take a look at the relationship between two
variables. n order to visualize that, it is better to plot a
\textbf{scatter plot.} (Plotting graphs will be covered in detail a
separate notebook).

\emph{Brief note on Scatterplots:}

In this kind of plot, each observation corresponds to one dot: the
horizontal location of the dot plots the value of the observation on one
variable, and the vertical location displays its value on the other
variable. In many situations you don't really have a clear opinion about
what the causal relationship is (e.g., does A cause B, or does B cause
A, or does some other variable C controls both A and B). If that's the
case, it doesn't really matter which variable you plot on the x-axis and
which one you plot on the y-axis. However, in many situations you do
have a pretty strong idea which variable you think is most likely to be
causal, or at least you have some suspicions in that direction. If so,
then it's conventional to plot the \textbf{cause} variable on the
\textbf{x-axis}, and the \textbf{effect} variable on the
\textbf{y-axis}.

Suppose our goal is to draw a scatterplot displaying the relationship
between the amount of sleep that Dan gets (dan.sleep) and how grumpy she
is the next day (dan.grump). \emph{Do you suspect a causal relationship
here?}

A simple way to plot these scatter plots is to use the scatterplot()
function in the car package.

Let's load the package and get started.

\begin{Shaded}
\begin{Highlighting}[]
\FunctionTok{install.packages}\NormalTok{(}\StringTok{"car"}\NormalTok{,}\AttributeTok{repos =} \StringTok{"http://cran.us.r{-}project.org"}\NormalTok{)}
\end{Highlighting}
\end{Shaded}

\begin{verbatim}
## Installing package into 'C:/Users/hp/AppData/Local/R/win-library/4.4'
## (as 'lib' is unspecified)
\end{verbatim}

\begin{verbatim}
## package 'car' successfully unpacked and MD5 sums checked
## 
## The downloaded binary packages are in
##  C:\Users\hp\AppData\Local\Temp\RtmpeSIJEr\downloaded_packages
\end{verbatim}

\begin{Shaded}
\begin{Highlighting}[]
\FunctionTok{install.packages}\NormalTok{(}\StringTok{"Rcpp"}\NormalTok{,}\AttributeTok{repos =} \StringTok{"http://cran.us.r{-}project.org"}\NormalTok{)}
\end{Highlighting}
\end{Shaded}

\begin{verbatim}
## Installing package into 'C:/Users/hp/AppData/Local/R/win-library/4.4'
## (as 'lib' is unspecified)
\end{verbatim}

\begin{verbatim}
## package 'Rcpp' successfully unpacked and MD5 sums checked
\end{verbatim}

\begin{verbatim}
## Warning: cannot remove prior installation of package 'Rcpp'
\end{verbatim}

\begin{verbatim}
## Warning in file.copy(savedcopy, lib, recursive = TRUE): problem copying
## C:\Users\hp\AppData\Local\R\win-library\4.4\00LOCK\Rcpp\libs\x64\Rcpp.dll to
## C:\Users\hp\AppData\Local\R\win-library\4.4\Rcpp\libs\x64\Rcpp.dll: Permission
## denied
\end{verbatim}

\begin{verbatim}
## Warning: restored 'Rcpp'
\end{verbatim}

\begin{verbatim}
## 
## The downloaded binary packages are in
##  C:\Users\hp\AppData\Local\Temp\RtmpeSIJEr\downloaded_packages
\end{verbatim}

\begin{Shaded}
\begin{Highlighting}[]
\FunctionTok{library}\NormalTok{(car)}
\end{Highlighting}
\end{Shaded}

\begin{verbatim}
## Loading required package: carData
\end{verbatim}

\begin{verbatim}
## 
## Attaching package: 'car'
\end{verbatim}

\begin{verbatim}
## The following object is masked from 'package:dplyr':
## 
##     recode
\end{verbatim}

\begin{verbatim}
## The following object is masked from 'package:purrr':
## 
##     some
\end{verbatim}

\begin{verbatim}
## The following object is masked from 'package:psych':
## 
##     logit
\end{verbatim}

\begin{Shaded}
\begin{Highlighting}[]
\FunctionTok{scatterplot}\NormalTok{( dan.grump }\SpecialCharTok{\textasciitilde{}}\NormalTok{ dan.sleep, }\AttributeTok{data =}\NormalTok{ parenthood, }\AttributeTok{regLine =} \ConstantTok{TRUE}\NormalTok{, }\AttributeTok{smooth =} \ConstantTok{FALSE}\NormalTok{)}
\end{Highlighting}
\end{Shaded}

\includegraphics{Module3_Nb2_files/figure-latex/unnamed-chunk-6-1.pdf}

\begin{Shaded}
\begin{Highlighting}[]
\NormalTok{scatterplot}
\end{Highlighting}
\end{Shaded}

\begin{verbatim}
## function (x, ...) 
## {
##     UseMethod("scatterplot")
## }
## <bytecode: 0x0000020151e46258>
## <environment: namespace:car>
\end{verbatim}

\begin{Shaded}
\begin{Highlighting}[]
\CommentTok{\#Plot a scatter plot for baby.sleep and dan.grump variables}
\FunctionTok{scatterplot}\NormalTok{(baby.sleep}\SpecialCharTok{\textasciitilde{}}\NormalTok{dan.grump, }\AttributeTok{data=}\NormalTok{parenthood,}\AttributeTok{regLine=}\ConstantTok{TRUE}\NormalTok{,}\AttributeTok{smooth=}\ConstantTok{FALSE}\NormalTok{ )}
\end{Highlighting}
\end{Shaded}

\includegraphics{Module3_Nb2_files/figure-latex/unnamed-chunk-7-1.pdf}

Just by plain observation and comparison, you can see that the
relationship is qualitatively the same in both cases: more sleep equals
less grump! However, it's also pretty obvious that the relationship
between dan.sleep and dan.grump is stronger than the relationship
between baby.sleep and dan.grump.

But what about the plot between baby.sleep and dan.sleep?

\begin{Shaded}
\begin{Highlighting}[]
\CommentTok{\#Plot baby sleep and dan sleep here}
\FunctionTok{scatterplot}\NormalTok{(baby.sleep}\SpecialCharTok{\textasciitilde{}}\NormalTok{dan.sleep, }\AttributeTok{data =}\NormalTok{ parenthood, }\AttributeTok{regLine=}\ConstantTok{TRUE}\NormalTok{,}\AttributeTok{smooth=}\ConstantTok{FALSE}\NormalTok{)}
\end{Highlighting}
\end{Shaded}

\includegraphics{Module3_Nb2_files/figure-latex/unnamed-chunk-8-1.pdf}

Is the direction of this plot same as the earlier plots? What about
strength?

\subparagraph{Correlation coefficient}\label{correlation-coefficient}

In order to to quantitatively represent the relationships of strength
and direction we discussed above, we can use correlation coefficient.

The correlation coefficient (or Pearson's correlation coefficient)
between two variables X and Y (sometimes denoted
\emph{r\textsubscript{XY}} ) is a measure that varies from -1 to 1. When
\emph{r} = -1 it means that we have a perfect negative relationship, and
when \emph{r} = 1 it means we have a perfect positive relationship. When
\emph{r} = 0, there's no relationship at all.

Look at the plots for different \emph{r} values:

\begin{figure}
\centering
\includegraphics{C:/Users/hp/Documents/GitHub/BSE658/Module 3/Notebooks/fig4.png}
\caption{Correlation plots}
\end{figure}

\subparagraph{Covariance}\label{covariance}

The covariance between two variables X and Y is a generalisation of the
notion of the variance; it's a mathematically simple way of describing
the relationship between two variables:

\begin{align*}
 
 Cov (X, Y) = \frac{1}{N-1}\sum_{i=1}^{N} (X- \overline{X} ) (Y- \overline{Y} )  \\
 
 \end{align*}

Covariance can be understood as an ``average cross product'' between X
and Y . The covariance has the nice property that, if X and Y are
entirely unrelated, then the covariance is exactly zero. If it is
positive, then the covariance is also positive; and if the relationship
is negative then the covariance is also negative. But as it has weird
units (try seeing for yourself), it si difficult to interpret and
therefore we standardise the covariance, the exact same way that the
z-score standardises a raw score: by dividing by the standard deviation.
However, because we have two variables that contribute to the
covariance, the standardisation only works if we divide by both standard
deviations.

This is what we call as the correlation coefficent, \emph{r}:

\begin{align*}

 r~XY~ = \frac{Cov(X,Y)}{\sigma_{X} \sigma_{Y}}

\end{align*}

This way, covariance properties are retained and it also becomes
interpretable.

Now let's check out how to code this using cor().

\begin{Shaded}
\begin{Highlighting}[]
\FunctionTok{cor}\NormalTok{(}\AttributeTok{x =}\NormalTok{ parenthood}\SpecialCharTok{$}\NormalTok{dan.sleep, }\AttributeTok{y =}\NormalTok{ parenthood}\SpecialCharTok{$}\NormalTok{dan.grump)}
\end{Highlighting}
\end{Shaded}

\begin{verbatim}
## [1] -0.903384
\end{verbatim}

\begin{Shaded}
\begin{Highlighting}[]
\FunctionTok{correlate}\NormalTok{(parenthood)}
\end{Highlighting}
\end{Shaded}

\begin{verbatim}
## 
## CORRELATIONS
## ============
## - correlation type:  pearson 
## - correlations shown only when both variables are numeric
## 
##            dan.sleep baby.sleep dan.grump    day
## dan.sleep          .      0.628    -0.903 -0.098
## baby.sleep     0.628          .    -0.566 -0.010
## dan.grump     -0.903     -0.566         .  0.076
## day           -0.098     -0.010     0.076      .
\end{verbatim}

\begin{Shaded}
\begin{Highlighting}[]
\CommentTok{\#Try giving the entire dataframe \textquotesingle{}parenthood\textquotesingle{} as input in cor()}
\end{Highlighting}
\end{Shaded}

What did you find?

\subparagraph{What does r = 0.4 mean?}\label{what-does-r-0.4-mean}

It really depends on what you want to use the data for, and on how
strong the correlations in your field tend to be.

\begin{figure}
\centering
\includegraphics{C:/Users/hp/Documents/GitHub/BSE658/Module 3/Notebooks/fig5.png}
\caption{Correlation coefficient interpretation table}
\end{figure}

Now let's take a look at this data called ``Anscombe's Quartet''

\begin{Shaded}
\begin{Highlighting}[]
\FunctionTok{load}\NormalTok{( }\StringTok{"anscombesquartet.Rdata"}\NormalTok{ )}
\FunctionTok{cor}\NormalTok{( X1, Y1 )}
\end{Highlighting}
\end{Shaded}

\begin{verbatim}
## [1] 0.8164205
\end{verbatim}

\begin{Shaded}
\begin{Highlighting}[]
\FunctionTok{cor}\NormalTok{( X2, Y2 )}
\end{Highlighting}
\end{Shaded}

\begin{verbatim}
## [1] 0.8162365
\end{verbatim}

\begin{Shaded}
\begin{Highlighting}[]
\FunctionTok{cor}\NormalTok{ (X3, Y3)}
\end{Highlighting}
\end{Shaded}

\begin{verbatim}
## [1] 0.8162867
\end{verbatim}

\begin{Shaded}
\begin{Highlighting}[]
\FunctionTok{cor}\NormalTok{ (X4, Y4)}
\end{Highlighting}
\end{Shaded}

\begin{verbatim}
## [1] 0.8165214
\end{verbatim}

Were the correlation coefficients same?

Now try plotting them.

\begin{Shaded}
\begin{Highlighting}[]
\FunctionTok{scatterplot}\NormalTok{(x1}\SpecialCharTok{\textasciitilde{}}\NormalTok{y1,}\AttributeTok{data=}\NormalTok{anscombe,}\AttributeTok{regLine=}\ConstantTok{FALSE}\NormalTok{,}\AttributeTok{smooth=}\ConstantTok{TRUE}\NormalTok{)}
\end{Highlighting}
\end{Shaded}

\includegraphics{Module3_Nb2_files/figure-latex/unnamed-chunk-11-1.pdf}

\begin{Shaded}
\begin{Highlighting}[]
\FunctionTok{scatterplot}\NormalTok{(x2}\SpecialCharTok{\textasciitilde{}}\NormalTok{y2,}\AttributeTok{data=}\NormalTok{anscombe,}\AttributeTok{regLine=}\ConstantTok{FALSE}\NormalTok{,}\AttributeTok{smooth=}\ConstantTok{TRUE}\NormalTok{)}
\end{Highlighting}
\end{Shaded}

\includegraphics{Module3_Nb2_files/figure-latex/unnamed-chunk-11-2.pdf}

\begin{Shaded}
\begin{Highlighting}[]
\FunctionTok{scatterplot}\NormalTok{(x3}\SpecialCharTok{\textasciitilde{}}\NormalTok{y3,}\AttributeTok{data=}\NormalTok{anscombe,}\AttributeTok{regLine=}\ConstantTok{FALSE}\NormalTok{,}\AttributeTok{smooth=}\ConstantTok{TRUE}\NormalTok{)}
\end{Highlighting}
\end{Shaded}

\includegraphics{Module3_Nb2_files/figure-latex/unnamed-chunk-11-3.pdf}

\begin{Shaded}
\begin{Highlighting}[]
\FunctionTok{scatterplot}\NormalTok{(x4}\SpecialCharTok{\textasciitilde{}}\NormalTok{y4,}\AttributeTok{data=}\NormalTok{anscombe,}\AttributeTok{regLine=}\ConstantTok{FALSE}\NormalTok{,}\AttributeTok{smooth=}\ConstantTok{TRUE}\NormalTok{)}
\end{Highlighting}
\end{Shaded}

\includegraphics{Module3_Nb2_files/figure-latex/unnamed-chunk-11-4.pdf}

Therefore, remember to always look at the scatterplot before attaching
any interpretation to the data!

If we have to properly define the role of Pearson's coefficient, we can
say that it actually measures the strength of the linear relationship
between two variables. In other words, it gives a measure of the extent
to which the data all tend to fall on a single, perfectly straight line.

\subparagraph{Spearman's Rank Order Correlation
Coefficient}\label{spearmans-rank-order-correlation-coefficient}

But let's take a look at another dataset and find correlation between
its variables.

\begin{Shaded}
\begin{Highlighting}[]
\FunctionTok{setwd}\NormalTok{(}\StringTok{"C:/Users/hp/Documents/GitHub/BSE658/Module 3/Notebooks/"}\NormalTok{)}
\FunctionTok{load}\NormalTok{( }\StringTok{"effort.Rdata"}\NormalTok{ )}
\NormalTok{effort}
\end{Highlighting}
\end{Shaded}

\begin{verbatim}
##    hours grade
## 1      2    13
## 2     76    91
## 3     40    79
## 4      6    14
## 5     16    21
## 6     28    74
## 7     27    47
## 8     59    85
## 9     46    84
## 10    68    88
\end{verbatim}

\begin{Shaded}
\begin{Highlighting}[]
\FunctionTok{cor}\NormalTok{( effort}\SpecialCharTok{$}\NormalTok{hours, effort}\SpecialCharTok{$}\NormalTok{grade )}
\end{Highlighting}
\end{Shaded}

\begin{verbatim}
## [1] 0.909402
\end{verbatim}

If you plot this -

\begin{Shaded}
\begin{Highlighting}[]
\FunctionTok{scatterplot}\NormalTok{(effort}\SpecialCharTok{$}\NormalTok{hours, effort}\SpecialCharTok{$}\NormalTok{grade, }\AttributeTok{regLine =} \ConstantTok{TRUE}\NormalTok{, }\AttributeTok{smooth =} \ConstantTok{FALSE}\NormalTok{)}
\end{Highlighting}
\end{Shaded}

\includegraphics{Module3_Nb2_files/figure-latex/unnamed-chunk-13-1.pdf}

The correlation \emph{r} = 0.91 we get above doe snot represent the
actual relationship the plot is depicting. What we're looking for is
something that captures the fact that there is a perfect \textbf{ordinal
relationship} here. That is, if student 1 works more hours than student
2, then we can guarantee that student 1 will get the better grade.

If we're looking for ordinal relationships, all we have to do is treat
the data as if it were ordinal scale! So, instead of measuring effort in
terms of ``hours worked'', let's rank all 10 of the students in order of
hours worked. That is, student 1 did the least work out of anyone (2
hours) so they get the lowest rank (rank = 1). Student 4 was the next
laziest, putting in only 6 hours of work in over the whole semester, so
they get the next lowest rank (rank = 2).

\begin{Shaded}
\begin{Highlighting}[]
\NormalTok{hours.rank }\OtherTok{\textless{}{-}} \FunctionTok{rank}\NormalTok{( effort}\SpecialCharTok{$}\NormalTok{hours )   }\CommentTok{\# rank students by hours worked}
\NormalTok{grade.rank }\OtherTok{\textless{}{-}} \FunctionTok{rank}\NormalTok{( effort}\SpecialCharTok{$}\NormalTok{grade )   }\CommentTok{\# rank students by grade received}

\CommentTok{\#Now try cor() function for these}
\FunctionTok{cor}\NormalTok{( hours.rank, grade.rank )}
\end{Highlighting}
\end{Shaded}

\begin{verbatim}
## [1] 1
\end{verbatim}

Now the correlation coefficient we get is different from the Perason's
correlation coefficient \emph{r} we got earlier. This new correlation
coefficient that we got is called `\textbf{Spearman's Correlation
Coefficient}', denoted by \(\rho\).

\begin{Shaded}
\begin{Highlighting}[]
\CommentTok{\#Execute this and compare with the correlation coefficient we got above}
\FunctionTok{cor}\NormalTok{( effort}\SpecialCharTok{$}\NormalTok{hours, effort}\SpecialCharTok{$}\NormalTok{grade, }\AttributeTok{method =} \StringTok{"spearman"}\NormalTok{)}
\end{Highlighting}
\end{Shaded}

\begin{verbatim}
## [1] 1
\end{verbatim}

\subparagraph{Handling missing values}\label{handling-missing-values}

We've seen in earlier lectures that there could be missing values in
data which are represented by \texttt{NA} in R. One easy way to remove
them is using \texttt{na.rm\ =\ TRUE} as argument in many functions.

But what if we have missing values in a dataframe where we have to find
correlations across variables.

Let's look at such a dataset.

\begin{Shaded}
\begin{Highlighting}[]
\FunctionTok{load}\NormalTok{( }\StringTok{"parenthood2.Rdata"}\NormalTok{ )}
\FunctionTok{print}\NormalTok{( parenthood2 )}
\end{Highlighting}
\end{Shaded}

\begin{verbatim}
##     dan.sleep baby.sleep dan.grump day
## 1        7.59         NA        56   1
## 2        7.91      11.66        60   2
## 3        5.14       7.92        82   3
## 4        7.71       9.61        55   4
## 5        6.68       9.75        NA   5
## 6        5.99       5.04        72   6
## 7        8.19      10.45        53   7
## 8        7.19       8.27        60   8
## 9        7.40       6.06        NA   9
## 10       6.58       7.09        71  10
## 11       6.49      11.68        72  11
## 12       6.27       6.13        65  12
## 13       5.95       7.83        74  13
## 14       6.65       5.60        67  14
## 15       6.41       6.03        66  15
## 16       6.33       8.19        69  16
## 17       6.30       6.38        73  17
## 18       8.47      11.11        52  18
## 19       7.21       5.51        61  19
## 20       7.53       6.69        53  20
## 21       8.00       9.74        54  21
## 22       7.35       9.02        63  22
## 23       6.86       6.44        74  23
## 24       7.86       9.43        56  24
## 25       4.86       3.46        82  25
## 26       5.87       6.32        72  26
## 27       8.40       7.95        NA  27
## 28         NA       7.69        66  28
## 29       7.21       7.45        60  29
## 30       6.99         NA        67  30
## 31       8.17       7.95        44  31
## 32       7.85         NA        53  32
## 33       6.27       4.70        76  33
## 34       8.66       8.52        41  34
## 35       4.98       4.70        86  35
## 36       6.19       8.32        60  36
## 37       6.41       9.38        63  37
## 38       4.84       4.18        89  38
## 39       7.03       5.98        61  39
## 40       7.66       9.29        57  40
## 41       7.51         NA        59  41
## 42       7.92      10.54        60  42
## 43       8.12      11.78        48  43
## 44       7.47      11.60        53  44
## 45       7.99      11.35        50  45
## 46       5.44       5.63        72  46
## 47       8.16       6.98        57  47
## 48       7.62       6.03        60  48
## 49       5.87       4.66        70  49
## 50       9.00       9.81        46  50
## 51       8.31      12.07        58  51
## 52       6.71       7.57        68  52
## 53       7.43      11.35        58  53
## 54       5.90         NA        71  54
## 55       8.52       8.29        52  55
## 56       6.03         NA        74  56
## 57       7.29         NA        59  57
## 58       7.32       8.59        59  58
## 59       6.88       7.82        67  59
## 60       6.22       7.18        67  60
## 61       6.94       8.29        61  61
## 62       7.01      11.08        64  62
## 63         NA       6.46        61  63
## 64         NA       3.25        61  64
## 65         NA       9.74        54  65
## 66       7.82       8.75        62  66
## 67       8.14      11.75        52  67
## 68       7.27       9.31        64  68
## 69         NA       7.73        65  69
## 70       7.55       8.68        NA  70
## 71       7.38       9.77        57  71
## 72       7.73       9.71        59  72
## 73       5.32         NA        79  73
## 74       7.86      10.18        53  74
## 75       6.35       9.28        NA  75
## 76       7.11         NA        61  76
## 77       5.45       6.38        82  77
## 78       7.80       9.20        68  78
## 79       7.13       8.20        67  79
## 80       8.35      10.16        54  80
## 81       6.93       8.95        53  81
## 82         NA       6.80        62  82
## 83       8.66       8.34        50  83
## 84       5.09       6.25        NA  84
## 85       4.91         NA        NA  85
## 86       7.03       9.09        62  86
## 87       7.02      10.42        64  87
## 88         NA       8.89        57  88
## 89       8.15       9.43        54  89
## 90       5.88       6.79        NA  90
## 91         NA       6.91        78  91
## 92       6.66       6.05        63  92
## 93       6.85         NA        59  93
## 94       5.57       8.62        74  94
## 95       5.16       7.84        76  95
## 96         NA       5.89        79  96
## 97       7.77       9.77        51  97
## 98       5.38       6.97        82  98
## 99       7.02       6.56        55  99
## 100      6.45       7.93        74 100
\end{verbatim}

\begin{Shaded}
\begin{Highlighting}[]
\FunctionTok{describe}\NormalTok{( parenthood2 ) }
\end{Highlighting}
\end{Shaded}

\begin{verbatim}
##            vars   n  mean    sd median trimmed   mad   min    max range  skew
## dan.sleep     1  91  6.98  1.02   7.03    7.02  1.13  4.84   9.00  4.16 -0.33
## baby.sleep    2  89  8.11  2.05   8.20    8.13  2.28  3.25  12.07  8.82 -0.09
## dan.grump     3  92 63.15  9.85  61.00   62.66 10.38 41.00  89.00 48.00  0.38
## day           4 100 50.50 29.01  50.50   50.50 37.06  1.00 100.00 99.00  0.00
##            kurtosis   se
## dan.sleep     -0.73 0.11
## baby.sleep    -0.59 0.22
## dan.grump     -0.31 1.03
## day           -1.24 2.90
\end{verbatim}

\begin{Shaded}
\begin{Highlighting}[]
\CommentTok{\#Check how many missing values are there for each variable {-} compare the values in \textquotesingle{}n\textquotesingle{} with the number of days.}
\end{Highlighting}
\end{Shaded}

Now, let's try finding correlations for this dataframe.

\begin{Shaded}
\begin{Highlighting}[]
\FunctionTok{cor}\NormalTok{(parenthood2)}
\end{Highlighting}
\end{Shaded}

\begin{verbatim}
##            dan.sleep baby.sleep dan.grump day
## dan.sleep          1         NA        NA  NA
## baby.sleep        NA          1        NA  NA
## dan.grump         NA         NA         1  NA
## day               NA         NA        NA   1
\end{verbatim}

In order top overcome this problem, we can use \texttt{use} as an
argument in the cor() function. Try out the following.

\begin{Shaded}
\begin{Highlighting}[]
\FunctionTok{cor}\NormalTok{(parenthood2, }\AttributeTok{use =} \StringTok{"complete.obs"}\NormalTok{)}
\end{Highlighting}
\end{Shaded}

\begin{verbatim}
##              dan.sleep baby.sleep   dan.grump         day
## dan.sleep   1.00000000  0.6394985 -0.89951468  0.06132891
## baby.sleep  0.63949845  1.0000000 -0.58656066  0.14555814
## dan.grump  -0.89951468 -0.5865607  1.00000000 -0.06816586
## day         0.06132891  0.1455581 -0.06816586  1.00000000
\end{verbatim}

\begin{Shaded}
\begin{Highlighting}[]
\FunctionTok{cor}\NormalTok{(parenthood2, }\AttributeTok{use =} \StringTok{"pairwise.complete.obs"}\NormalTok{)}
\end{Highlighting}
\end{Shaded}

\begin{verbatim}
##              dan.sleep  baby.sleep    dan.grump          day
## dan.sleep   1.00000000  0.61472303 -0.903442442 -0.076796665
## baby.sleep  0.61472303  1.00000000 -0.567802669  0.058309485
## dan.grump  -0.90344244 -0.56780267  1.000000000  0.005833399
## day        -0.07679667  0.05830949  0.005833399  1.000000000
\end{verbatim}

When we choose \texttt{use\ =\ "complete.obs"}, R will completely ignore
all cases (i.e., all rows in our parenthood2 data frame) that have any
missing values at all. For eg., if you choose use = ``complete.obs'' R
will ignore that row completely: that is, even when it's trying to
calculate the correlation between dan.sleep and dan.grump, observation 1
will be ignored, because the value of baby.sleep is missing for that
observation.

Whereas when we set \texttt{use\ =\ "pairwise.complete.obs"} R only
looks at the variables that it's trying to correlate when determining
what to drop. So, for instance, since the only missing value for
observation 1 of parenthood2 is for baby.sleep R will only drop
observation 1 when baby.sleep is one of the variables involved: and so R
keeps observation 1 when trying to correlate dan.sleep and dan.grump.

The above operation can also be performed by another function called
\texttt{correlate()} in \texttt{lsr} package. \#displays pairwise
spearman correlations Try it out.

\begin{Shaded}
\begin{Highlighting}[]
\CommentTok{\#Try correlate() for parenthood2 here}
\FunctionTok{correlate}\NormalTok{(parenthood2,}\AttributeTok{corr.method =} \StringTok{"pearson"}\NormalTok{)}
\end{Highlighting}
\end{Shaded}

\begin{verbatim}
## 
## CORRELATIONS
## ============
## - correlation type:  pearson 
## - correlations shown only when both variables are numeric
## 
##            dan.sleep baby.sleep dan.grump    day
## dan.sleep          .      0.615    -0.903 -0.077
## baby.sleep     0.615          .    -0.568  0.058
## dan.grump     -0.903     -0.568         .  0.006
## day           -0.077      0.058     0.006      .
\end{verbatim}

\emph{Reference : Chapter 5, D. Navarro}

That's all folks!

\end{document}
